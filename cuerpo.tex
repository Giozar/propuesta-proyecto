\pagestyle{fancy}
\section{Introducción}

AironTools es una empresa mexicana con más de una década de experiencia en la producción y comercialización de herramientas industriales. A lo largo de su trayectoria ha consolidado una base de clientes recurrentes en el sector industrial mécanico y automotriz. Sin embargo, actualmente enfrenta importantes desafíos operativos derivados del uso de procesos manuales y herramientas no integradas para la gestión interna y la atención a clientes.

La gestión de servicios técnicos, inventario, comunicación interna y control de información se realiza mediante hojas de cálculo y formatos físicos, lo que ha derivado en dificultades para la trazabilidad de los servicios, errores en el registro de información, pérdida de datos importantes para la toma de decisiones, y una experiencia limitada para el cliente, al no contar con un sistema que le permita dar seguimiento a sus solicitudes de manera eficiente.

Estas deficiencias afectan directamente la eficiencia operativa, la competitividad y la capacidad de crecimiento de AironTools frente a otras empresas del sector que ya operan con sistemas digitales robustos e interconectados. En este contexto, se propone el diseño e implementación de un sistema de gestión empresarial que permita digitalizar y automatizar los procesos clave de la organización.

El objetivo del proyecto es desarrollar una plataforma integral que centralice la información, mejore la trazabilidad de los servicios ofrecidos, optimice el control de inventario y facilite la comunicación entre las distintas áreas operativas. Este sistema estará alineado con los flujos de trabajo actuales de la empresa, y se construirá bajo una arquitectura escalable que garantice su adaptabilidad y crecimiento a futuro.

Como beneficio, se espera una mejora sustancial en los tiempos de respuesta, una administración más eficiente de los recursos, una atención al cliente más ágil y profesional, y un fortalecimiento de la capacidad competitiva de AironTools en el mercado de herramientas industriales. El presente proyecto se enmarca dentro de la modalidad de experiencia profesional, y responde a una necesidad real detectada en la operación de la empresa.

\section{Justificación}

La propuesta de este proyecto surge de una necesidad concreta identificada en la operación diaria de la empresa AironTools. En la actualidad, la falta de un sistema de gestión empresarial digitalizado ha derivado en múltiples dificultades operativas, entre las que destacan una atención deficiente al cliente, escasa trazabilidad de los servicios técnicos, desorganización en el control de inventario y una gestión interna basada en registros manuales. Esta situación ha limitado significativamente la capacidad de la empresa para escalar sus operaciones y mantenerse competitiva frente a organizaciones del mismo sector que ya han implementado soluciones tecnológicas robustas.

Los registros actuales de clientes, servicios, inventario y empleados se realizan mediante formatos físicos o archivos aislados, lo cual genera errores frecuentes, pérdida de información, duplicidad de datos y demoras en la atención. Estas deficiencias impactan de manera directa en la eficiencia del personal, en la capacidad de respuesta y en la imagen profesional que la empresa proyecta a sus clientes.

Frente a este panorama, la implementación de un sistema de gestión empresarial integral representa una solución estratégica. Este sistema permitirá centralizar en una sola plataforma la información relacionada con empleados, clientes, servicios y productos, lo que facilitará la trazabilidad del flujo de trabajo desde la solicitud de un servicio hasta la entrega final. Además, la automatización de notificaciones y la digitalización de los procesos permitirán establecer una comunicación más eficiente con los clientes, otorgándoles información puntual y transparente sobre el avance de sus solicitudes.

Otra ventaja clave de esta solución es la optimización del control de inventario. A través de funciones como reportes automático, será posible reducir errores, evitar datos faltantes críticos en documentaciones técnicas y anticipar necesidades operativas. A su vez, los reportes generados por el sistema facilitarán la toma de decisiones basadas en datos reales y actualizados, incrementando la capacidad de análisis y planificación de la empresa.

En conjunto, esta propuesta tecnológica contribuirá a mejorar la productividad interna, profesionalizar la operación diaria y elevar la calidad del servicio ofrecido a los clientes. También permitirá documentar formalmente los procesos internos, asegurando su replicabilidad y continuidad. La digitalización de estas funciones no solo responde a una necesidad interna urgente, sino que posiciona a AironTools como una empresa moderna, eficiente y con visión de crecimiento sostenible en un mercado altamente competitivo.

\section{Objetivo general}
Diseñar un sistema digital de gestión empresarial para AironTools que automatice y centralice los procesos internos de atención al cliente, servicios técnicos, control de inventario y comunicación organizacional, con el fin de mejorar la eficiencia operativa, la trazabilidad de los servicios y la calidad del servicio ofrecido.

\section{Objetivos específicos}
\begin{enumerate}
	\item Desarrollar un módulo de gestión de usuarios que permita registrar, organizar y asignar roles diferenciados al personal de AironTools según sus funciones operativas mediante permisos.

	\item Implementar un módulo de gestión de clientes que facilite el registro, seguimiento del historial de atención y mejora en la comunicación con los clientes.

	\item Automatizar el flujo completo de los servicios técnicos, desde el ingreso del equipo hasta su entrega, integrando notificaciones en cada etapa del proceso.

	\item Diseñar un módulo de control de inventario que permita registrar, editar y consultar productos con sus respectivos detalles, funcionando como catálogo actualizado de herramientas.
\end{enumerate}

\section{Trabajos relacionados}

A continuación se presentan ocho trabajos que abordan problemáticas similares a las identificadas en el presente proyecto. Se analiza su propósito, contexto, solución implementada y resultados obtenidos, con el fin de establecer un marco de referencia que oriente el diseño del sistema propuesto para AironTools.

\subsection{Elaboración de un sistema de gestión por procesos aplicado a una empresa dedicada a la distribución y comercialización de productos de ferretería y materiales de construcción \cite{Vargas2018}.}

Este proyecto integrador fue desarrollado en la Escuela Superior Politécnica del Litoral, Ecuador, con el objetivo de mejorar la eficiencia operativa de una empresa ferretera mediante la estandarización de sus procesos. La organización enfrentaba problemas de control interno, duplicidad de funciones y bajo aprovechamiento de recursos. Para atender esta situación, se diseñó un sistema de gestión basado en procesos, levantando y documentando operaciones clave como recepción, almacenaje, facturación y distribución. Se aplicaron herramientas como análisis FODA, flujogramas, metodología 5W+1H, diagrama de Ishikawa e indicadores de desempeño. Como resultado, se estructuró una propuesta organizacional orientada a optimizar la toma de decisiones, fortalecer el control administrativo y elevar la competitividad en el mercado.

\subsection{Desarrollo de un sistema de gestión por procesos para empresas de servicios de ingeniería y construcción orientadas a la industria \cite{Munoz2018}.}

Esta tesis de maestría tuvo como finalidad mejorar la eficiencia organizacional de la empresa CDM S.A., dedicada a servicios industriales de ingeniería y construcción, mediante el diseño de un sistema de gestión por procesos. La empresa presentaba deficiencias en la estandarización de actividades, fallos de comunicación entre áreas y ausencia de documentación formal. La propuesta se basó en teorías administrativas contemporáneas y en la norma ISO 9001:2015, integrando procesos estratégicos, operativos y de apoyo, con herramientas como mapas de procesos, diagramas de flujo, políticas, indicadores y mecanismos de mejora continua. El resultado fue un modelo estructurado que facilita la documentación, el control y la mejora de las operaciones clave.

\subsection{Diseño de un sistema de gestión basado en procesos \cite{Jacome2016}.}

Esta tesis de maestría, desarrollada en la Universidad Andina Simón Bolívar, propuso un sistema de gestión basado en procesos para una empresa de tecnología que enfrentaba desorganización operativa y baja eficiencia administrativa. Con el fin de mejorar la rentabilidad y el control interno, se estructuró una solución fundamentada en normas ISO, análisis detallado de procesos, definición de indicadores de desempeño y documentación formal. El resultado fue un modelo que fortaleció la claridad organizacional y facilitó la toma de decisiones mediante una gestión alineada a objetivos estratégicos.

\subsection{Proceso Administrativo y Gestión Empresarial en COPROABAS, Jinotega \cite{Flores15}.}

Esta tesis de maestría tuvo como propósito mejorar el desempeño organizacional de la cooperativa COPROABAS mediante el fortalecimiento de su gestión administrativa. La investigación identificó deficiencias en la estructuración organizativa, ausencia de manuales de funciones y falta de cultura administrativa. A partir de un enfoque descriptivo y transversal, se evaluó la aplicación del proceso administrativo y se propusieron mejoras orientadas a optimizar la planificación, organización, dirección y control dentro de la cooperativa.

\subsection{Principales causas del fracaso en la implementación de un ERP: el caso de una PyME en México \cite{Delgado2015}.}

Esta tesis de maestría desarrollada en la UNAM tuvo como finalidad identificar las causas que provocan el fracaso en la implementación de sistemas ERP en pequeñas y medianas empresas mexicanas. Mediante el método Delphi aplicado a expertos, se detectaron deficiencias en la capacitación, selección de personal y planeación estratégica. El estudio evidenció una limitada cultura tecnológica y desinterés gerencial como factores críticos, y concluyó con recomendaciones orientadas a mejorar la preparación y ejecución de futuros proyectos de ERP en el entorno PyME.

\subsection{Planeación Estratégica y Nuevos Proyectos en Empresa Propia \cite{Patino19}.}

Este proyecto terminal, desarrollado en la Universidad ICESI, tuvo como objetivo consolidar la planeación estratégica de una empresa apícola en fase inicial. La propuesta abordó la falta de estructura organizacional mediante el análisis FODA, el modelo Canvas y herramientas de diagnóstico interno. Se definieron la misión, visión y objetivos estratégicos, acompañados de proyecciones financieras, con el fin de orientar su crecimiento competitivo dentro del mercado colombiano.

\subsection{La gestión empresarial como factor clave de desarrollo de las spin-offs universitarias \cite{Rodeiro2012}.}

Este artículo, publicado en *Observatorio de la Economía Latinoamericana*, estudia las causas del bajo rendimiento de las spin-offs universitarias en Galicia, España, a pesar de su origen en entornos de alta investigación. Mediante un análisis basado en encuestas aplicadas a responsables de estas empresas, se identificaron como principales obstáculos la escasa experiencia en gestión, la débil estructura organizativa y las limitaciones financieras. El trabajo concluye con la recomendación de fortalecer las capacidades administrativas desde las etapas iniciales del emprendimiento.

\subsection{Gestión Empresarial y Desarrollo \cite{Reyes12}.}

Esta tesis doctoral plantea un enfoque sistémico y estratégico de la gestión empresarial, integrando el análisis del entorno interno, las relaciones inmediatas de la organización y el contexto macroeconómico. Aunque se trata de un estudio teórico, propone modelos referenciales y recomendaciones prácticas orientadas al fortalecimiento organizacional, con énfasis en la sostenibilidad, el desarrollo humano y la adaptación a contextos complejos.

% NOTA: Actualizar también la tabla de comparación si es necesario.


\begin{longtable}{m{.05\paperwidth} *{2}{m{.33\paperwidth}} @{}}
	\caption{Comparación cualitativa de los trabajos relacionados con el proyecto.}
	\label{table:trabajosRelacionados}\\
	\hline
	\textbf{Ref.} & \textbf{Similitudes} & \textbf{Diferencias} \\
	\hline
	\endfirsthead
	
	\multicolumn{3}{c}{\textbf{Continuación de la Tabla \ref{table:trabajosRelacionados}}} \\
	\hline
	\textbf{Ref.} & \textbf{Similitudes} & \textbf{Diferencias} \\
	\hline
	\endhead
	\hline
	\endlastfoot

\cite{Vargas2018} &
\begin{itemize}[topsep=0pt,itemsep=0pt,parsep=0pt,partopsep=0pt,leftmargin=*]
	\item Ambos proyectos aplican la gestión por procesos como metodología base para el rediseño organizacional.
	\item Se emplean herramientas estratégicas como FODA, flujogramas e indicadores para mejorar procesos internos.
	\item Se enfocan en optimizar la eficiencia operativa y mejorar la toma de decisiones empresariales.
\end{itemize} &
\begin{itemize}[topsep=0pt,itemsep=0pt,parsep=0pt,partopsep=0pt,leftmargin=*]
	\item El proyecto de Vargas se limita a una propuesta conceptual sin implementación digital; mi sistema está completamente desarrollado e implementado con tecnologías modernas.
	\item Mi proyecto incorpora automatización, trazabilidad y comunicación digital en tiempo real, ausentes en el trabajo analizado.
	\item La propuesta se centra en un enfoque contable y administrativo, mientras que mi solución abarca también soporte técnico, inventario, servicios y arquitectura multitenant.
\end{itemize} \\
\midrule

\cite{Munoz2018} &
\begin{itemize}[topsep=0pt,itemsep=0pt,parsep=0pt,partopsep=0pt,leftmargin=*]
	\item Ambos trabajos proponen un sistema de gestión enfocado en procesos clave, estratégicos y de soporte.
	\item Se identifican deficiencias internas como punto de partida para justificar la propuesta de solución.
	\item Incluyen el diseño de indicadores, políticas, documentación y mejora continua como ejes del sistema.
\end{itemize} &
\begin{itemize}[topsep=0pt,itemsep=0pt,parsep=0pt,partopsep=0pt,leftmargin=*]
	\item El proyecto de Muñoz fue aplicado a una empresa específica (CDM S.A.) sin desarrollo tecnológico o software.
	\item Mi propuesta incluye una solución tecnológica funcional con desarrollo de software usando NestJS, React y MongoDB.
	\item La tesis se enfoca en la estructuración teórica y documental, mientras que mi proyecto automatiza procesos y permite trazabilidad digital.
\end{itemize} \\
\midrule

\cite{Jacome2016} &
\begin{itemize}[topsep=0pt,itemsep=0pt,parsep=0pt,partopsep=0pt,leftmargin=*]
	\item Ambos proyectos están centrados en sistemas de gestión orientados a procesos.
	\item Se propone estructurar operaciones críticas con enfoque organizacional.
	\item Se reconoce la importancia de medir y controlar el desempeño.
\end{itemize} &
\begin{itemize}[topsep=0pt,itemsep=0pt,parsep=0pt,partopsep=0pt,leftmargin=*]
	\item La tesis es documental y analítica; mi proyecto desarrolla un sistema funcional con tecnologías modernas.
	\item Mi solución incluye comunicación en tiempo real, automatización y despliegue en la nube.
	\item Está enfocado en el sector industrial de herramientas, no en ventas tecnológicas.
\end{itemize} \\
\midrule

\cite{Flores15} &
\begin{itemize}[topsep=0pt,itemsep=0pt,parsep=0pt,partopsep=0pt,leftmargin=*]
	\item Ambos trabajos analizan los elementos del proceso administrativo (planeación, organización, dirección y control).
	\item Se busca fortalecer la gestión empresarial para mejorar el desempeño y sostenibilidad de la organización.
	\item Se identifican problemas derivados de una gestión deficiente y se proponen alternativas de solución.
\end{itemize} &
\begin{itemize}[topsep=0pt,itemsep=0pt,parsep=0pt,partopsep=0pt,leftmargin=*]
	\item La tesis se basa en un análisis cualitativo, mientras que mi proyecto desarrolla una solución digital implementada en una empresa real.
	\item Mi trabajo aplica tecnologías modernas como NestJS, React y MongoDB, en contraste con el enfoque tradicional de evaluación administrativa.
	\item La propuesta planteada en mi proyecto incluye automatización, trazabilidad y comunicación digital, aspectos ausentes en el trabajo de Flores.
\end{itemize} \\
\midrule

\cite{Delgado2015} &
\begin{itemize}[topsep=0pt,itemsep=0pt,parsep=0pt,partopsep=0pt,leftmargin=*]
	\item Ambos proyectos abordan problemáticas empresariales relacionadas con la eficiencia organizacional y el uso de herramientas tecnológicas.
	\item Coinciden en identificar la gestión adecuada de procesos como factor clave para mejorar el desempeño empresarial.
	\item Ambos consideran la importancia de los recursos humanos y la estructura organizacional como elementos fundamentales en la implementación de soluciones.
\end{itemize} &
\begin{itemize}[topsep=0pt,itemsep=0pt,parsep=0pt,partopsep=0pt,leftmargin=*]
	\item Mi proyecto propone una solución técnica funcional basada en herramientas como NestJS, React y MongoDB, mientras que el trabajo de Delgado es un análisis metodológico de causas de fracaso en la implementación de ERP.
	\item El enfoque de la tesis de Delgado es predominantemente teórico y orientado al diagnóstico, mientras que el mío es de tipo aplicado con resultados tangibles en una empresa real.
	\item Mi propuesta aborda la automatización y trazabilidad en procesos empresariales, mientras que Delgado se enfoca en generar conocimiento para prevenir errores comunes.
\end{itemize} \\
\midrule

\cite{Patino19} &
\begin{itemize}[topsep=0pt,itemsep=0pt,parsep=0pt,partopsep=0pt,leftmargin=*]
	\item Ambos proyectos se desarrollaron para resolver una problemática empresarial real.
	\item Ambos utilizan principios de planeación estratégica para estructurar el funcionamiento de una organización.
	\item Se enfocan en mejorar la eficiencia operativa y competitividad de una empresa.
\end{itemize} &
\begin{itemize}[topsep=0pt,itemsep=0pt,parsep=0pt,partopsep=0pt,leftmargin=*]
	\item El proyecto de Kopec S.A.S. se basa en estrategias organizacionales, mientras que el de AironTools implementa una solución tecnológica concreta.
	\item El trabajo de Kopec no contempla desarrollo de software ni arquitectura técnica; AironTools desarrolla un sistema digital completo.
	\item El proyecto de AironTools incluye infraestructura DevOps, IA y soporte multitenant, aspectos ausentes en Kopec.
\end{itemize} \\
\midrule

\cite{Rodeiro2012} &
\begin{itemize}[topsep=0pt,itemsep=0pt,parsep=0pt,partopsep=0pt,leftmargin=*]
	\item Ambos trabajos reconocen la importancia de la gestión empresarial como clave para la sostenibilidad y crecimiento organizacional.
	\item Se abordan barreras estructurales y organizativas que deben atenderse para mejorar el desempeño de una empresa.
	\item Se destaca la necesidad de fortalecer procesos de gestión desde etapas tempranas del desarrollo empresarial.
\end{itemize} &
\begin{itemize}[topsep=0pt,itemsep=0pt,parsep=0pt,partopsep=0pt,leftmargin=*]
	\item El artículo se centra en el análisis de spin-offs universitarias mediante encuestas y datos estadísticos, mientras que en mi proyecto se desarrolla e implementa una solución tecnológica real en una empresa del sector industrial.
	\item En el trabajo analizado se identifican debilidades en capacidades administrativas, mientras que en mi proyecto se crean herramientas específicas para solucionar dichas deficiencias, como módulos de control, automatización, reportes y trazabilidad.
	\item El artículo propone recomendaciones generales sobre gestión y financiación; en cambio, mi propuesta incluye la integración de tecnologías como NestJS, React, MongoDB y despliegue en AWS.
\end{itemize} \\
\midrule

\cite{Reyes12} &
\begin{itemize}[topsep=0pt,itemsep=0pt,parsep=0pt,partopsep=0pt,leftmargin=*]
	\item Ambos trabajos abordan la gestión empresarial como medio para mejorar el desempeño organizacional.
	\item Reconocen la importancia de factores internos, del entorno inmediato y del contexto macroeconómico.
	\item Promueven la integración de capacidades humanas, innovación y sostenibilidad para lograr organizaciones más eficientes.
\end{itemize} &
\begin{itemize}[topsep=0pt,itemsep=0pt,parsep=0pt,partopsep=0pt,leftmargin=*]
	\item El proyecto de Reyes es de tipo académico-conceptual, mientras que el tuyo es una solución aplicada y funcional implementada en una empresa real.
	\item Tu proyecto utiliza herramientas tecnológicas específicas (NestJS, React, MongoDB), mientras que el de Reyes se mantiene en el plano teórico sin desarrollo de software.
	\item El enfoque del proyecto de Reyes está centrado en el desarrollo humano y la inclusión social, mientras que el tuyo se enfoca en la eficiencia operativa y trazabilidad empresarial.
\end{itemize} \\
\bottomrule
\end{longtable}

	

\section{Descripción técnica}

El sistema propuesto para AironTools es una aplicación web de gestión empresarial compuesta por módulos funcionales independientes que automatizan y centralizan los procesos clave de la organización. Está orientado a mejorar la atención al cliente, la administración de servicios técnicos, el control de productos y la coordinación interna. Todos los módulos interactúan bajo una arquitectura escalable y mantenible, permitiendo su crecimiento progresivo.

\subsection*{Arquitectura general del sistema}

El sistema sigue una arquitectura modular cliente-servidor, donde el backend y el frontend se desarrollan de forma desacoplada. La comunicación entre ambos se realiza mediante APIs RESTful. Esta estructura permite una mayor flexibilidad, mantenibilidad y facilidad de integración futura con nuevas funcionalidades.

La Figura~\ref{fig:arquitectura} muestra una vista general de la arquitectura del sistema con sus principales componentes.

\begin{figure}[H]
	\centering
	\includegraphics[width=0.9\textwidth]{sistema.png}
	\caption{Arquitectura general del sistema propuesto para AironTools.}
	\label{fig:arquitectura}
\end{figure}

\subsection*{Módulo de usuarios y roles}

Este módulo permite registrar empleados, asignar roles diferenciados y gestionar permisos de acceso al sistema. Incluye funcionalidades de autenticación y autorización para garantizar la seguridad del sistema, además de una interfaz de administración para el rol de superadministrador.

\subsection*{Módulo de clientes}

Permite registrar tanto a clientes individuales como a empresas, asociando datos de contacto e historial de servicios recibidos. Este módulo se integra con el sistema de notificaciones para mantener informados a los clientes durante cada etapa del servicio técnico.

\subsection*{Módulo de servicios técnicos}

Automatiza el flujo completo de atención a servicios como reparación, mantenimiento, demostración y cotización. El sistema registra desde el ingreso del equipo hasta su diagnóstico, reparación, validación final y entrega al cliente. Se incluyen notificaciones automáticas por cambios de estado.

\begin{figure}[H]
	\centering
	\includegraphics[width=0.9\textwidth]{secuencia-sistema.png}
	\caption{Diagrama de secuencia del flujo de atención a servicios técnicos.}
	\label{fig:secuencia}
\end{figure}

\begin{figure}[H]
	\centering
	\includegraphics[width=0.9\textwidth]{casos-uso-sistema.png}
	\caption{Diagrama de casos de uso asociados a los módulos funcionales del sistema.}
	\label{fig:casosuso}
\end{figure}

\subsection*{Módulo de productos}

Este módulo permite registrar, editar y consultar productos e insumos disponibles para la venta o uso interno. Los productos se organizan jerárquicamente en familias, categorías y subcategorías, con información como nombre, precio, descripción, imágenes y diagramas técnicos. Actualmente funciona como un catálogo visual centralizado, integrable con los servicios y clientes.

\subsection*{Módulo de notificaciones y comunicación interna}

El sistema integra notificaciones por correo electrónico según el estado del servicio. También se contempla un sistema básico de comunicación en tiempo real mediante WebSockets para uso interno entre áreas operativas, con opción de extenderse a chat interno completo.

\subsection*{Resumen visual de módulos e interacciones}

La Figura~\ref{fig:modulos} muestra un esquema visual de los módulos implementados y cómo se comunican entre sí para cumplir con los objetivos del sistema.

\begin{figure}[H]
	\centering
	\includegraphics[width=0.9\textwidth]{componentes-sistema.png}
	\caption{Módulos principales del sistema de gestión y sus interacciones.}
	\label{fig:modulos}
\end{figure}

	
\section{Especificación técnica}

El sistema será accesible mediante una aplicación web dividida en backend y frontend independientes. Los módulos funcionales operan de forma integrada a través de APIs RESTful, alineados con los flujos reales de la empresa. La arquitectura es modular, escalable y mantenible.

\subsection*{Tecnologías utilizadas}

\begin{itemize}
	\item \textbf{Backend:} NestJS con TypeScript sobre Node.js. Arquitectura modular compuesta por controladores, servicios, repositorios, DTOs y validaciones centralizadas usando \texttt{class-validator}. Esta estructura permite separar responsabilidades y mantener el sistema escalable y mantenible.
	\item \textbf{Frontend:} React.js con TypeScript. Arquitectura basada en \textbf{Screaming Architecture} y \textbf{Clean Architecture}, lo cual permite que la estructura del proyecto refleje claramente sus funcionalidades, promoviendo una alta cohesión y bajo acoplamiento entre componentes.
	\item \textbf{Base de datos:} MongoDB, con soporte para múltiples empresas (multitenant) mediante aislamiento lógico de datos.
	\item \textbf{Interfaz de usuario:} HTML5 y CSS3, apoyado con Figma para diseño de prototipos y TalwindCSS para componentes accesibles y responsivos.
	\item \textbf{Comunicación:} WebSockets para mensajería interna en tiempo real y actualizaciones automáticas de estados.
	\item \textbf{Correo electrónico:} Protocolo SMTP para el envío automático de notificaciones operativas a usuarios internos y clientes.
	\item \textbf{Control de versiones e integración continua:} Git para versionamiento y GitHub Actions para automatización del pipeline CI/CD.
	\item \textbf{Contenerización y despliegue:} Uso de Docker para encapsular servicios, con despliegue en AWS (EC2 para la aplicación, S3 para archivos y Lambda para procesos automatizados).
\end{itemize}


\subsection*{Alcance funcional}

El proyecto contempla como mínimo la implementación de los siguientes módulos:

\begin{itemize}
	\item Gestión de empleados con autenticación y roles.
	\item Registro y administración de clientes individuales y empresariales.
	\item Flujo completo de servicios técnicos con seguimiento automatizado.
	\item Catálogo digital de productos e insumos.
	\item Notificaciones por correo y sistema de comunicación interna.
	\item Respaldo automático configurable y generación de reportes técnicos.
	\item Soporte multitenant para múltiples compañías, según la empresa seleccionada al iniciar sesión.
\end{itemize}

\subsection*{Criterios de finalización}

El sistema se considerará funcional cuando:

\begin{itemize}
	\item Cada módulo esté implementado y probado funcionalmente.
	\item Los casos de uso estén validados y documentados.
	\item El sistema esté desplegado en un entorno de pruebas funcional.
	\item Se incluya documentación técnica, manual de usuario y código fuente organizado.
	\item Se entregue una carpeta digital con el PDF final, código fuente comprimido y apéndices con el listado de código.
\end{itemize}


	
	Cada módulo será considerado finalizado cuando cumpla con los casos de uso definidos, haya sido probado funcionalmente y esté debidamente documentado. Además, el sistema deberá encontrarse desplegado en un entorno de pruebas funcional para su demostración.
	
	La Figura~\ref{fig:modulos} muestra un esquema general de los módulos definidos, y la Figura~\ref{fig:casosuso} representa los principales casos de uso considerados para la validación del sistema.
	
	\vspace{0.5cm}
	
	%Este texto SÍ debe incluirse para que la propuesta pueda ser aceptada.
	Al concluir el proyecto de integración se entregará a la Coordinación de Estudios de Ingeniería en Computación una carpeta digital que incluirá el reporte final del proyecto en un archivo PDF (sin restricciones)\footnote{Debe poder visualizarse sin solicitar contraseña}, el código fuente del proyecto en un archivo comprimido (sin restricciones)\footnote{Debe poder descomprimirse sin solicitar contraseña}. Además, la sección de apéndices del reporte final contendrá al menos un listado del código fuente desarrollado.


	\section{Calendario de actividades}

Las actividades a realizar durante el Trimestre 2025-Invierno en la UEA Proyecto de Integración de Ingeniería en Computación I (clave 1100113), con un valor de 18 créditos y una duración total de 198 horas, se presentan en la Tabla~\ref{table:calendarioActividades}.

\begin{longtable}{p{0.05\textwidth} p{0.4\textwidth} p{0.1\textwidth} p{0.35\textwidth}}
	\caption{Listado de actividades a realizar durante el Trimestre 2025-Invierno.}
  	\label{table:calendarioActividades}\\
	\toprule
	\textbf{No.} & \textbf{Actividad} & \textbf{Horas} & \textbf{Entregable} \\
	\hline
	\endfirsthead

	\multicolumn{4}{c}{\textbf{Continuación de la Tabla \ref{table:calendarioActividades}}}\\
	\hline
	\textbf{No.} & \textbf{Actividad} & \textbf{Horas} & \textbf{Entregable} \\
	\hline
	\endhead

	\hline
	\endlastfoot

	1 & Levantamiento de requerimientos y análisis del sistema actual en AironTools. & 20 & Documento de requerimientos \\
	\midrule

	2 & Diseño de arquitectura del sistema y definición de módulos. & 20 & Diagramas de arquitectura y diseño técnico \\
	\midrule

	3 & Desarrollo del módulo de autenticación y gestión de usuarios. & 25 & Módulo funcional con control de acceso y roles \\
	\midrule

	4 & Desarrollo del módulo de gestión de clientes. & 20 & Registro, historial y vista de clientes implementados \\
	\midrule

	5 & Desarrollo del módulo de servicios técnicos (flujo completo). & 25 & Módulo de flujo de servicio técnico automatizado \\
	\midrule

	6 & Desarrollo del módulo de inventario. & 20 & Registro, movimientos y alertas de inventario \\
	\midrule

	7 & Implementación de sistema de notificaciones y comunicación interna. & 20 & Notificaciones por correo, alertas internas y tareas asignadas \\
	\midrule

	8 & Pruebas funcionales e integración de los módulos. & 20 & Reporte de pruebas, casos de uso validados \\
	\midrule

	9 & Despliegue en entorno de pruebas y revisión técnica. & 15 & Sistema desplegado en servidor y documentación preliminar \\
	\midrule

	10 & Documentación técnica y elaboración de manual de usuario. & 13 & Manual de usuario, guía de instalación y documentación del sistema \\
	\bottomrule
\end{longtable}

\footnotetext{La planeación cubre el análisis, diseño, desarrollo, pruebas, despliegue y documentación del sistema, abarcando las 198 horas correspondientes a la UEA mencionada.}




\section{Factibilidad técnica y operativa}

\subsection{Factibilidad técnica}

El proyecto de desarrollo del sistema de gestión empresarial para AironTools es técnicamente viable, ya que el responsable del desarrollo cuenta con los conocimientos y habilidades necesarios para implementar los módulos definidos en el tiempo estipulado. Entre las competencias destacadas se encuentran:

\begin{itemize}
	\item Conocimientos avanzados en desarrollo web fullstack utilizando tecnologías como React.js, TypeScript y NestJS.
	\item Experiencia en diseño e implementación de arquitecturas modulares y sistemas empresariales.
	\item Habilidad en pruebas, documentación técnica y despliegue de sistemas.
\end{itemize}

Los recursos disponibles para el desarrollo y pruebas del sistema son los siguientes:

\begin{itemize}
	\item Equipos de cómputo con procesadores Intel Core i7, 16 GB de RAM y almacenamiento SSD.
	\item Acceso a entornos locales y servidores remotos para pruebas y despliegue del sistema.
	\item Conectividad de red estable para realizar pruebas multiusuario y simulaciones reales.
\end{itemize}

Las herramientas que se utilizarán durante el desarrollo incluyen:

\begin{itemize}
	\item \textbf{Frontend:} React.js con TypeScript, Figma para prototipado, y TalwindCSS para diseño de interfaz.
	\item \textbf{Backend:} NestJS con Node.js y MongoDB como base de datos.
	\item \textbf{Control de versiones:} Git y GitHub.
	\item \textbf{Contenedores y despliegue:} Docker, GitHub Actions para integración y entrega continua.
\end{itemize}

No se requieren recursos físicos adicionales ni licencias de software especiales, ya que todas las herramientas son de uso libre o ya están disponibles.

\subsection{Factibilidad operativa}

El sistema propuesto presenta una alta factibilidad operativa dentro de la empresa AironTools, por las siguientes razones:

\begin{itemize}
	\item \textbf{Adaptabilidad a los procesos actuales:} El sistema está alineado con los flujos de trabajo reales de la empresa, por lo que su adopción no requerirá una reestructuración significativa.
	
	\item \textbf{Aceptación organizacional:} El proyecto cuenta con el respaldo directo del jefe de área, el Ing. Víctor Benjamín Aguilar Orocio, responsable de la implementación en la empresa.
	
	\item \textbf{Capacitación del personal:} Se prevé una estrategia de introducción y formación progresiva, asegurando que los empleados comprendan y utilicen el sistema eficazmente.
	
	\item \textbf{Facilidad de uso y soporte:} La interfaz estará diseñada con criterios de usabilidad, y el sistema incluirá funciones de ayuda y comunicación interna que facilitarán su soporte continuo.
	
	\item \textbf{Extensibilidad y mantenimiento:} Gracias a su arquitectura modular, el sistema podrá adaptarse a futuras necesidades o integrarse con nuevas tecnologías sin comprometer su estabilidad.
\end{itemize}

\vspace{0.5cm}
\section{Estimación de costos}

La presente sección muestra una estimación comercial del capital necesario para el desarrollo del sistema de gestión empresarial propuesto, considerando tanto los recursos técnicos como el valor del trabajo intelectual involucrado. Se han incluido costos asociados a infraestructura, herramientas, servicios y personal. Esta estimación representa una aproximación realista en caso de que el proyecto se desee escalar o comercializar.

La estimación de costos del proyecto se presenta en la Tabla~\ref{table:tablaCostos}.

\begin{longtable}{m{6.5cm} m{4.5cm} m{4cm}}
	\caption{Estimación de costos del proyecto.}
  	\label{table:tablaCostos}\\
  	\toprule
	\textbf{Descripción} & \textbf{Costo unitario (MXN)} & \textbf{Costo total (MXN)} \\
	\hline
	\endfirsthead

	\multicolumn{3}{c}{\textbf{Continuación de la Tabla \ref{table:tablaCostos}}}\\
	\hline
	\textbf{Descripción} & \textbf{Costo unitario (MXN)} & \textbf{Costo total (MXN)} \\
	\hline
	\endhead

	\hline
	\endlastfoot

Infraestructura de servidores (entorno de pruebas y producción) & \$3,000.00 $\times$ 1 mes & \$9,000.00 (3 meses) \\
\midrule

Servicios en la nube (almacenamiento, correos, CI/CD) & \$2,000.00 $\times$ 1 mes & \$6,000.00 (3 meses) \\
\midrule

Trabajo de desarrollo (fullstack)~\cite{Desarrollador} & \$25,000.00 $\times$ 1 mes & \$75,000.00 (3 meses) \\
\midrule

Diseño UI/UX y prototipado de interfaces & \$10,000.00 $\times$ 1 mes & \$30,000.00 (3 meses) \\
\midrule

Capacitación y soporte técnico interno & \$5,000.00 $\times$ 1 mes & \$15,000.00 (3 meses) \\
\midrule

Pruebas, validación y documentación & \$4,000.00 $\times$ 1 mes & \$12,000.00 (3 meses) \\
\midrule

\textbf{Costo total estimado} & — & \textbf{\$147,000.00} \\
\bottomrule
\end{longtable}

\footnotetext{Los valores fueron obtenidos con base en referencias de mercado, experiencia previa en proyectos similares y validación con el responsable directo en la empresa. Incluyen costos de infraestructura, servicios, diseño, desarrollo y soporte.}
