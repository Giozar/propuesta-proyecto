\section{Introducción}

AironTools es una empresa mexicana con más de una década de experiencia en la producción y comercialización de herramientas industriales. A lo largo del tiempo, ha consolidado una base de clientes recurrentes en el sector metalmecánico y de servicios técnicos. Sin embargo, actualmente enfrenta importantes desafíos operativos derivados del uso de procesos manuales y sistemas no integrados para la gestión interna y atención a clientes.

La gestión de servicios técnicos, inventario, comunicación y control de información se realiza mediante hojas de cálculo o formatos físicos, lo cual ha generado una serie de problemáticas operativas tales como:

\begin{itemize}
    \item Dificultades en el seguimiento y trazabilidad de servicios técnicos.
    \item Registro manual y propenso a errores en la información de clientes, empleados e inventario.
    \item Comunicación ineficiente entre áreas operativas.
    \item Pérdida de información relevante para la toma de decisiones.
    \item Limitada experiencia del cliente en cuanto al seguimiento de sus solicitudes.
\end{itemize}

Estas deficiencias impactan directamente en la eficiencia operativa, en la satisfacción del cliente y en la capacidad de crecimiento de la empresa frente a competidores que ya operan con sistemas digitales robustos.

Ante este contexto, se propone el diseño e implementación de un sistema de gestión empresarial integral, enfocado en digitalizar y automatizar los procesos clave de AironTools. Este sistema permitirá centralizar la información, mejorar la trazabilidad de los servicios, optimizar el control de inventario y establecer una comunicación más efectiva entre los diferentes departamentos de la empresa.

La solución tecnológica será desarrollada con base en las necesidades específicas de AironTools, garantizando su alineación con los flujos de trabajo actuales y considerando una arquitectura escalable que permita su crecimiento progresivo. Este proyecto surge como una experiencia profesional dentro de la empresa, con la finalidad de mejorar su competitividad, eficiencia interna y nivel de servicio al cliente.

\section{Justificación}

La propuesta de este proyecto surge de una necesidad concreta detectada dentro de la operación diaria de AironTools. Al no contar con un sistema de gestión empresarial digitalizado, la empresa enfrenta problemas críticos en áreas como la atención al cliente, la trazabilidad de servicios, el control de inventario y la gestión interna.

Actualmente, los registros de clientes, servicios, inventario y empleados se realizan manualmente, lo que conlleva a errores frecuentes, pérdida de información, duplicidad de datos y demoras en la atención. Esta situación ha limitado la capacidad de la empresa para escalar sus operaciones y ofrecer un servicio eficiente y profesional.

Implementar un sistema de gestión empresarial permitirá:

\begin{itemize}
    \item Centralizar la información de empleados, clientes, servicios y productos en una sola plataforma.
    \item Optimizar la trazabilidad del flujo de servicios técnicos, desde la solicitud hasta la entrega.
    \item Automatizar notificaciones y comunicación con clientes en cada etapa del servicio.
    \item Facilitar el control de inventario mediante alertas de stock, reportes automáticos y seguimiento de movimientos.
    \item Mejorar la eficiencia operativa mediante la digitalización de procesos clave.
    \item Fortalecer la toma de decisiones mediante reportes e indicadores basados en datos confiables.
\end{itemize}

Además, la digitalización de estos procesos permitirá que la empresa se mantenga competitiva frente a otras del sector que ya han adoptado herramientas tecnológicas para optimizar sus operaciones. Esta propuesta tiene un impacto directo en la calidad del servicio, la productividad interna y la capacidad de la empresa para responder ágilmente a las necesidades del mercado.

Finalmente, el desarrollo de este sistema permitirá documentar formalmente los procesos internos de AironTools, profesionalizar la operación diaria y brindar una experiencia de usuario moderna, confiable y transparente a sus clientes.

\section{Objetivos}

\subsection{Objetivo General}

Diseñar e implementar un sistema de gestión empresarial digital para AironTools que permita automatizar y centralizar los procesos internos de atención al cliente, gestión de servicios técnicos, control de inventario y comunicación organizacional, con el fin de mejorar la eficiencia operativa, la trazabilidad y la calidad del servicio.

\subsection{Objetivos Específicos}

\begin{itemize}
    \item Desarrollar un módulo de gestión de usuarios que permita registrar, organizar y asignar roles diferenciados al personal de AironTools según sus funciones operativas.
    
    \item Implementar un módulo de gestión de clientes que facilite el registro, seguimiento de historial, contacto y servicios solicitados, mejorando la comunicación con el cliente.
    
    \item Diseñar e implementar un flujo automatizado de servicios técnicos, desde el ingreso del equipo hasta la entrega final, integrando notificaciones en cada etapa.
    
    \item Desarrollar un módulo de control de inventario que registre entradas, salidas, movimientos de productos y alertas por niveles críticos de stock.
\end{itemize}

\section{Trabajos Relacionados}

A continuación se presentan algunos proyectos y estudios relevantes que han servido como referencia para el desarrollo de esta propuesta, ya sea por su similitud en el enfoque tecnológico, el contexto empresarial o las soluciones planteadas.

\subsection{Digitalización de procesos en empresas manufactureras \cite{Garcia07}}

Este proyecto terminal expone una solución enfocada en la optimización de procesos dentro de empresas manufactureras mediante sistemas digitales. El autor plantea la necesidad de transformar procesos manuales en digitales para aumentar la eficiencia, mejorar el control interno y facilitar el crecimiento. Su propuesta destaca la importancia de integrar herramientas tecnológicas en contextos industriales similares al de AironTools.

\subsection{Automatización en servicios técnicos con enfoque operativo \cite{Nin1992}}

El estudio analiza la implementación de sistemas de automatización en empresas que ofrecen servicios técnicos especializados. Plantea la necesidad de digitalizar el ciclo completo de servicio para reducir errores, mejorar la trazabilidad y elevar la satisfacción del cliente. Es una referencia clave por su enfoque operativo y estructura del flujo técnico.

\subsection{Transformación digital en PyMEs industriales \cite{Rodeiro2012}}

Este artículo explora cómo las pequeñas y medianas empresas del sector industrial pueden abordar la transformación digital para mejorar su competitividad. Estudia los principales factores que afectan la adopción tecnológica y ofrece estrategias de implementación con bajo costo y alto impacto. AironTools comparte características con este tipo de organizaciones.

\subsection{Sistemas de gestión de inventario en el sector de herramientas \cite{Flores2015}}

Este trabajo de maestría describe la implementación de un sistema de gestión de inventario en una empresa cooperativa del sector ferretero. Se destacan los beneficios de tener visibilidad en tiempo real del stock, con reducción en pérdidas y mejora en el control operativo. La propuesta para AironTools se inspira en este enfoque específico de productos físicos.

\subsection{Importancia de la comunicación interna digitalizada \cite{Reyes2012}}

El estudio plantea que una comunicación interna efectiva dentro de una organización mejora la productividad y la coordinación entre departamentos. Propone el uso de plataformas digitales como solución a los problemas de fragmentación en empresas de tamaño medio. Este principio será implementado mediante un módulo de mensajería y notificaciones en el sistema para AironTools.

\subsection{Mejora de la experiencia del cliente mediante sistemas integrados \cite{Patino2019}}

Esta investigación de grado evalúa cómo la planificación estratégica y la integración de sistemas digitales influyen en la percepción del cliente y la fidelización. A través de un caso práctico en una empresa de servicios, se identifican elementos clave para mejorar la comunicación y seguimiento. AironTools podría beneficiarse de aplicar estos principios en su flujo de servicios técnicos.

\begin{longtable}{|p{0.24\textwidth}|p{0.23\textwidth}|p{0.23\textwidth}|p{0.23\textwidth}|}
    \hline
    \textbf{Referencia} & \textbf{Ventajas} & \textbf{Desventajas} & \textbf{Similitudes con el Proyecto} \\
    \hline
    \endfirsthead
    
    \hline
    \textbf{Referencia} & \textbf{Ventajas} & \textbf{Desventajas} & \textbf{Similitudes con el Proyecto} \\
    \hline
    \endhead
    
    \hline
    \cite{Garcia07} &
    \begin{itemize}
        \item Alineado al sector industrial.
        \item Uso de herramientas digitales para procesos internos.
        \item Análisis de eficiencia operativa.
    \end{itemize} &
    \begin{itemize}
        \item Enfoque general, sin especialización en servicios técnicos.
    \end{itemize} &
    \begin{itemize}
        \item Proceso de digitalización industrial.
        \item Mejora del control interno.
    \end{itemize} \\
    \hline
    
    \cite{Nin1992} &
    \begin{itemize}
        \item Automatización completa del flujo técnico.
        \item Análisis detallado de trazabilidad.
    \end{itemize} &
    \begin{itemize}
        \item Tecnología desactualizada.
        \item Escaso enfoque en clientes.
    \end{itemize} &
    \begin{itemize}
        \item Flujo de servicio técnico automatizado.
        \item Seguimiento de procesos.
    \end{itemize} \\
    \hline
    
    \cite{Rodeiro2012} &
    \begin{itemize}
        \item Contexto PyME relevante.
        \item Estrategias de adopción tecnológica.
    \end{itemize} &
    \begin{itemize}
        \item Poca aplicación práctica directa.
    \end{itemize} &
    \begin{itemize}
        \item Enfoque en transformación digital en PyMEs.
    \end{itemize} \\
    \hline
    
    \cite{Flores2015} &
    \begin{itemize}
        \item Enfoque específico en herramientas.
        \item Control detallado del stock.
    \end{itemize} &
    \begin{itemize}
        \item Alcance limitado a inventario.
    \end{itemize} &
    \begin{itemize}
        \item Gestión de inventario.
        \item Productos industriales físicos.
    \end{itemize} \\
    \hline
    
    \cite{Reyes2012} &
    \begin{itemize}
        \item Mejora de comunicación organizacional.
        \item Énfasis en eficiencia de procesos.
    \end{itemize} &
    \begin{itemize}
        \item Sin implementación tecnológica clara.
    \end{itemize} &
    \begin{itemize}
        \item Mejora de la comunicación interna en la empresa.
    \end{itemize} \\
    \hline
    
    \cite{Patino2019} &
    \begin{itemize}
        \item Orientación al cliente.
        \item Medición de la experiencia del usuario.
    \end{itemize} &
    \begin{itemize}
        \item Contexto diferente (empresa propia).
    \end{itemize} &
    \begin{itemize}
        \item Satisfacción del cliente y seguimiento del servicio.
    \end{itemize} \\
    \hline
    \caption{Comparación de trabajos relacionados con la propuesta de sistema de gestión empresarial para AironTools.}
    \label{tabla:trabajosRelacionados}
    \end{longtable}

    \section{Descripción Técnica}

El sistema de gestión empresarial propuesto para AironTools está diseñado como una aplicación web moderna que permite automatizar procesos clave como la atención al cliente, la gestión de servicios técnicos, el control de inventario y la comunicación interna. Se desarrollará utilizando tecnologías actuales que aseguran escalabilidad, seguridad y mantenimiento eficiente.

\subsection{Arquitectura del Sistema}

Se utilizará una arquitectura basada en microservicios, lo cual permitirá escalar módulos de forma independiente, mejorar la tolerancia a fallos y facilitar la implementación modular. Cada servicio podrá operar de forma autónoma y comunicarse con los demás a través de APIs RESTful seguras.

\subsection{Módulos Principales del Sistema}

\begin{itemize}
    \item \textbf{Módulo de Usuarios y Roles}: Gestiona el acceso al sistema mediante autenticación y autorización. Soporta distintos perfiles (administrador, ventas, técnico, etc.).
    
    \item \textbf{Módulo de Clientes}: Permite registrar y consultar datos de los clientes, sus servicios activos e historial.
    
    \item \textbf{Módulo de Servicios Técnicos}: Controla el flujo de atención a herramientas, desde su ingreso hasta la entrega. Incluye diagnósticos, cotizaciones, reparaciones y cierre.
    
    \item \textbf{Módulo de Inventario}: Administra el stock de herramientas, repuestos y productos, con alertas automáticas y reportes de movimientos.
    
    \item \textbf{Módulo de Notificaciones}: Informa al cliente sobre el estado de sus servicios mediante correo electrónico y alertas internas. También facilita la comunicación entre empleados.
\end{itemize}

\subsection{Tecnologías Utilizadas}

\textbf{Frontend:}
\begin{itemize}
    \item \textbf{React.js} con \textbf{TypeScript} para una interfaz interactiva y mantenible.
    \item \textbf{HTML5} y \textbf{CSS3} para estructura y estilos.
    \item \textbf{Axios} para comunicación con el backend.
    \item \textbf{Figma} y \textbf{Material UI} para diseño de interfaz (UI/UX).
\end{itemize}

\textbf{Backend:}
\begin{itemize}
    \item \textbf{NestJS} con \textbf{TypeScript}, basado en arquitectura modular.
    \item \textbf{MongoDB} como base de datos no relacional por su flexibilidad.
    \item \textbf{Node.js} como entorno de ejecución.
    \item \textbf{Jest} para pruebas automatizadas.
\end{itemize}

\textbf{Otros componentes:}
\begin{itemize}
    \item \textbf{Docker} para contenedores y despliegue.
    \item \textbf{GitHub Actions} para integración y despliegue continuo (CI/CD).
    \item \textbf{Visual Studio Code} como entorno de desarrollo.
\end{itemize}








\section{Especificación Técnica}

El desarrollo del sistema de gestión empresarial para AironTools se basa en un enfoque modular, escalable y centrado en los procesos críticos de la empresa. El sistema será accesible a través de una aplicación web y podrá ser utilizado por empleados autorizados desde distintos dispositivos.

\subsection{Alcance del Proyecto}

El proyecto incluye el diseño, desarrollo e implementación de los siguientes módulos funcionales:

\begin{itemize}
    \item Registro y gestión de empleados con roles diferenciados.
    \item Registro de clientes y su historial de servicios.
    \item Gestión completa del flujo de servicios técnicos.
    \item Control del inventario de herramientas y productos.
    \item Notificaciones automáticas para empleados y clientes.
    \item Comunicación interna por mensajería o correo empresarial.
    \item Generación de reportes operativos.
\end{itemize}

Cada módulo será desarrollado, probado e integrado de forma progresiva para asegurar una implementación ordenada.

\subsection{Tecnologías y Arquitectura}

El sistema utilizará una arquitectura cliente-servidor basada en microservicios. El frontend será desarrollado en \textbf{React.js} y \textbf{TypeScript}, mientras que el backend será implementado en \textbf{NestJS} sobre \textbf{Node.js}. Para el almacenamiento de datos se usará \textbf{MongoDB}, una base de datos NoSQL flexible y escalable.

El sistema se ejecutará en contenedores mediante \textbf{Docker}, y el flujo de integración y entrega continua será gestionado con \textbf{GitHub Actions}.

\subsection{Criterio de Finalización del Proyecto}

El proyecto se considerará finalizado cuando todos los módulos definidos hayan sido desarrollados, integrados y validados mediante pruebas funcionales. Además, deberá estar documentado correctamente y desplegado en un entorno de pruebas funcional.

\begin{itemize}
    \item Todos los módulos deben cumplir con los casos de uso definidos.
    \item Debe existir documentación técnica y manual de usuario.
    \item El sistema debe estar instalado en un servidor funcional o entorno en la nube.
    \item Se debe realizar una presentación funcional ante la Coordinación.
\end{itemize}

\textbf{Figura \ref{fig:componentes-sistema}} muestra la arquitectura general del sistema, y la \textbf{Figura \ref{fig:casos-uso}} representa los principales casos de uso que deben cumplirse para la validación del proyecto.

\vspace{0.5cm}

%Este texto SÍ debe incluirse para que la propuesta pueda ser aceptada.
Al concluir el proyecto de integración se entregará a la Coordinación de Estudios de Ingeniería en Computación una carpeta digital que incluirá el reporte final del proyecto en un archivo PDF (sin restricciones)\footnote{Debe poder visualizarse sin solicitar contraseña}, el código fuente de la aplicación en un archivo comprimido (sin restricciones)\footnote{Debe poder descomprimirse sin solicitar contraseña}. Además, la sección de apéndices del reporte final contendrá al menos un listado del código fuente desarrollado.

\section{Calendario de Actividades}

El desarrollo del proyecto se llevará a cabo durante el trimestre 2025-Invierno, como parte de la UEA "Proyecto de Integración de Ingeniería en Computación I" (clave 1100113), con un valor de 18 créditos y una duración de 198 horas.

A continuación se presenta el calendario de actividades, distribuido en 10 fases principales, cada una con su respectivo número de horas y entregables.

\begin{longtable}{p{0.05\textwidth} p{0.45\textwidth} p{0.1\textwidth} p{0.30\textwidth}}
  \label{table:calendarioActividades}\\
  \toprule
\textbf{No.} & \textbf{Actividad} & \textbf{Horas} & \textbf{Entregable} \\
\hline
\endfirsthead

\hline
\textbf{No.} & \textbf{Actividad} & \textbf{Horas} & \textbf{Entregable} \\
\hline
\endhead

\hline
\caption{Listado de actividades a realizar durante el trimestre 2025-Invierno.}
\endlastfoot

1 & Levantamiento de requerimientos y análisis del sistema actual en AironTools. & 20 & Documento de requerimientos \\
\midrule

2 & Diseño de arquitectura del sistema y definición de módulos. & 20 & Diagramas de arquitectura y diseño técnico \\
\midrule

3 & Desarrollo del módulo de autenticación y gestión de usuarios. & 25 & Módulo funcional con control de acceso y roles \\
\midrule

4 & Desarrollo del módulo de gestión de clientes. & 20 & Registro, historial y vista de clientes implementados \\
\midrule

5 & Desarrollo del módulo de servicios técnicos (flujo completo). & 25 & Módulo de flujo de servicio técnico automatizado \\
\midrule

6 & Desarrollo del módulo de inventario. & 20 & Registro, movimientos y alertas de inventario \\
\midrule

7 & Implementación de sistema de notificaciones y comunicación interna. & 20 & Notificaciones por correo, alertas internas y tareas asignadas \\
\midrule

8 & Pruebas funcionales e integración de los módulos. & 20 & Reporte de pruebas, casos de uso validados \\
\midrule

9 & Despliegue en entorno de pruebas y revisión técnica. & 15 & Sistema desplegado en servidor y documentación preliminar \\
\midrule

10 & Documentación técnica y elaboración de manual de usuario. & 13 & Manual de usuario, guía de instalación y documentación del sistema \\
\bottomrule
\end{longtable}

\footnotetext{Las actividades fueron diseñadas para cubrir el desarrollo completo del sistema propuesto durante el trimestre, incluyendo análisis, diseño, codificación, pruebas, despliegue y documentación final.}

\section{Factibilidad Técnica, Operativa y Estimación de Costos}

\subsection{Factibilidad Operativa}

La implementación del sistema de gestión empresarial presenta una alta factibilidad operativa en AironTools por las siguientes razones:

\begin{itemize}
    \item \textbf{Adaptabilidad a procesos existentes:} El sistema se diseñará según las necesidades reales y flujos actuales de la empresa.
    \item \textbf{Capacitación gradual del personal:} Se planea una estrategia progresiva de formación para cada área.
    \item \textbf{Soporte técnico disponible:} El responsable del proyecto dentro de la empresa garantizará la continuidad técnica.
    \item \textbf{Aprobación de dirección:} El proyecto cuenta con el respaldo directo del jefe de área: \textbf{Mtro. Víctor Benjamín Aguilar Orocio}.
\end{itemize}

\subsection{Factibilidad Técnica}

El proyecto es técnicamente viable debido a los siguientes elementos disponibles:

\subsubsection{Recursos de desarrollo y pruebas}
\begin{itemize}
    \item Equipos de cómputo con procesadores Intel i7, 16GB RAM y SSDs.
    \item Acceso a entornos locales y remotos para desarrollo y despliegue.
    \item Conectividad de red estable para pruebas multiusuario.
\end{itemize}

\subsubsection{Herramientas utilizadas}
\begin{itemize}
    \item \textbf{Frontend:} React.js + TypeScript, Figma, Material UI.
    \item \textbf{Backend:} NestJS, Node.js, MongoDB.
    \item \textbf{Control de versiones:} Git y GitHub.
    \item \textbf{Despliegue:} Docker, GitHub Actions.
\end{itemize}

\subsubsection{Conocimientos del desarrollador}
\begin{itemize}
    \item Experiencia en diseño de software empresarial.
    \item Conocimientos avanzados en desarrollo fullstack.
    \item Capacidad para realizar pruebas, documentación y despliegue.
\end{itemize}

\subsection{Estimación de Costos}

La siguiente tabla muestra una estimación aproximada de los costos asociados al desarrollo e implementación del sistema durante el trimestre.

\begin{longtable}{|p{0.5\textwidth}|p{0.2\textwidth}|p{0.2\textwidth}|}
\hline
\textbf{Rubro} & \textbf{Costo mensual (MXN)} & \textbf{Costo trimestral (MXN)} \\
\hline
\endfirsthead

\hline
\textbf{Rubro} & \textbf{Costo mensual (MXN)} & \textbf{Costo trimestral (MXN)} \\
\hline
\endhead

Infraestructura de servidores y pruebas & \$3,000 & \$9,000 \\
\hline
Licencias y servicios en la nube & \$2,000 & \$6,000 \\
\hline
Desarrollo de software (frontend y backend) & \$25,000 & \$75,000 \\
\hline
Diseño UI/UX y prototipos & \$10,000 & \$30,000 \\
\hline
Capacitación y soporte al personal & \$5,000 & \$15,000 \\
\hline
Pruebas de calidad y documentación & \$4,000 & \$12,000 \\
\hline
\textbf{Total estimado} & \textbf{\$49,000} & \textbf{\$147,000} \\
\hline
\caption{Estimación de costos para el desarrollo del sistema propuesto.}
\label{tabla:costos}
\end{longtable}

\footnotetext{Los costos fueron estimados con base en referencias de mercado y validación interna con el responsable del área. Incluyen desarrollo, pruebas, capacitación e infraestructura necesaria para el trimestre.}


\begin{figure}[H]
    \centering
    \includegraphics[width=0.8\textwidth]{componentes-sistema.png}
    \caption{Diagrama de Componentes del Sistema}
    \label{fig:componentes-sistema}
\end{figure}


\begin{figure}[H]
    \centering
    \includegraphics[width=0.8\textwidth]{casos-uso-sistema.png}
    \caption{Diagrama de Casos de Uso del Sistema}
    \label{fig:casos-uso}
\end{figure}


\begin{figure}[H]
    \centering
    \includegraphics[width=0.8\textwidth]{secuencia-sistema.png}
    \caption{Diagrama de Secuencia del Sistema}
    \label{fig:secuencia-sistema}
\end{figure}


\begin{figure}[H]
    \centering
    \includegraphics[width=0.8\textwidth]{sistema.png}
    \caption{Vista general del sistema propuesto}
    \label{fig:vista-general}
\end{figure}