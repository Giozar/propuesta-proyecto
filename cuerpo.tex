\pagestyle{fancy}
\section{Introducción}

AironTools es una empresa mexicana con más de una década de experiencia en la producción y comercialización de herramientas industriales. A lo largo de su trayectoria ha consolidado una base de clientes recurrentes en el sector metalmecánico y de servicios técnicos. Sin embargo, actualmente enfrenta importantes desafíos operativos derivados del uso de procesos manuales y herramientas no integradas para la gestión interna y la atención a clientes.

La gestión de servicios técnicos, inventario, comunicación interna y control de información se realiza mediante hojas de cálculo y formatos físicos, lo que ha derivado en dificultades para la trazabilidad de los servicios, errores en el registro de información, pérdida de datos importantes para la toma de decisiones, y una experiencia limitada para el cliente, al no contar con un sistema que le permita dar seguimiento a sus solicitudes de manera eficiente.

Estas deficiencias afectan directamente la eficiencia operativa, la competitividad y la capacidad de crecimiento de AironTools frente a otras empresas del sector que ya operan con sistemas digitales robustos e interconectados. En este contexto, se propone el diseño e implementación de un sistema de gestión empresarial que permita digitalizar y automatizar los procesos clave de la organización.

El objetivo del proyecto es desarrollar una plataforma integral que centralice la información, mejore la trazabilidad de los servicios ofrecidos, optimice el control de inventario y facilite la comunicación entre las distintas áreas operativas. Este sistema estará alineado con los flujos de trabajo actuales de la empresa, y se construirá bajo una arquitectura escalable que garantice su adaptabilidad y crecimiento a futuro.

Como beneficio, se espera una mejora sustancial en los tiempos de respuesta, una administración más eficiente de los recursos, una atención al cliente más ágil y profesional, y un fortalecimiento de la capacidad competitiva de AironTools en el mercado de herramientas industriales. El presente proyecto se enmarca dentro de la modalidad de experiencia profesional, y responde a una necesidad real detectada en la operación de la empresa.

\section{Justificación}

La propuesta de este proyecto surge de una necesidad concreta identificada en la operación diaria de la empresa AironTools. En la actualidad, la falta de un sistema de gestión empresarial digitalizado ha derivado en múltiples dificultades operativas, entre las que destacan una atención deficiente al cliente, escasa trazabilidad de los servicios técnicos, desorganización en el control de inventario y una gestión interna basada en registros manuales. Esta situación ha limitado significativamente la capacidad de la empresa para escalar sus operaciones y mantenerse competitiva frente a organizaciones del mismo sector que ya han implementado soluciones tecnológicas robustas.

Los registros actuales de clientes, servicios, inventario y empleados se realizan mediante formatos físicos o archivos aislados, lo cual genera errores frecuentes, pérdida de información, duplicidad de datos y demoras en la atención. Estas deficiencias impactan de manera directa en la eficiencia del personal, en la capacidad de respuesta y en la imagen profesional que la empresa proyecta a sus clientes.

Frente a este panorama, la implementación de un sistema de gestión empresarial integral representa una solución estratégica. Este sistema permitirá centralizar en una sola plataforma la información relacionada con empleados, clientes, servicios y productos, lo que facilitará la trazabilidad del flujo de trabajo desde la solicitud de un servicio hasta la entrega final. Además, la automatización de notificaciones y la digitalización de los procesos permitirán establecer una comunicación más eficiente con los clientes, otorgándoles información puntual y transparente sobre el avance de sus solicitudes.

Otra ventaja clave de esta solución es la optimización del control de inventario. A través de funciones como alertas de stock, reportes automáticos y seguimiento de movimientos, será posible reducir errores, evitar faltantes críticos y anticipar necesidades operativas. A su vez, los reportes generados por el sistema facilitarán la toma de decisiones basadas en datos reales y actualizados, incrementando la capacidad de análisis y planificación de la empresa.

En conjunto, esta propuesta tecnológica contribuirá a mejorar la productividad interna, profesionalizar la operación diaria y elevar la calidad del servicio ofrecido a los clientes. También permitirá documentar formalmente los procesos internos, asegurando su replicabilidad y continuidad. La digitalización de estas funciones no solo responde a una necesidad interna urgente, sino que posiciona a AironTools como una empresa moderna, eficiente y con visión de crecimiento sostenible en un mercado altamente competitivo.

\section{Objetivo general}
Diseñar un sistema digital de gestión empresarial para AironTools que automatice y centralice los procesos internos de atención al cliente, servicios técnicos, control de inventario y comunicación organizacional, con el fin de mejorar la eficiencia operativa, la trazabilidad de los servicios y la calidad del servicio ofrecido.

\section{Objetivos específicos}
\begin{enumerate}
	\item Desarrollar un módulo de gestión de usuarios que permita registrar, organizar y asignar roles diferenciados al personal de AironTools según sus funciones operativas.

	\item Implementar un módulo de gestión de clientes que facilite el registro, seguimiento del historial de atención y mejora en la comunicación con los clientes.

	\item Automatizar el flujo completo de los servicios técnicos, desde el ingreso del equipo hasta su entrega, integrando notificaciones en cada etapa del proceso.

	\item Diseñar un módulo de control de inventario que gestione entradas, salidas, movimientos de productos y alertas por niveles críticos de stock.
\end{enumerate}

\section{Trabajos relacionados}

A continuación se presentan seis trabajos que abordan problemáticas similares a las identificadas en el presente proyecto. Se analiza su propósito, contexto, solución implementada y resultados obtenidos, con el fin de establecer un marco de referencia que oriente el diseño del sistema propuesto para AironTools.

\subsection{Digitalización de procesos en empresas manufactureras \cite{Garcia07}}

Este proyecto terminal atendió el problema de la ineficiencia operativa en empresas manufactureras que aún empleaban procesos manuales para su gestión interna. Su objetivo fue facilitar el crecimiento empresarial mediante la digitalización de dichos procesos. La solución consistió en diseñar e implementar un sistema digital de gestión que integrara herramientas tecnológicas para mejorar el control interno. Como resultado, se logró una mayor eficiencia y trazabilidad en operaciones administrativas.

\subsection{Automatización en servicios técnicos con enfoque operativo \cite{Nin1992}}

El estudio abordó las fallas de trazabilidad y errores humanos en la atención de servicios técnicos. Para mejorar la calidad del servicio al cliente, se propuso automatizar el ciclo completo del proceso técnico. La solución consistió en aplicar sistemas computacionales que digitalizan cada etapa del servicio, desde el ingreso hasta la entrega del equipo. Se obtuvo como resultado una reducción significativa en errores y tiempos de atención.

\subsection{Transformación digital en PyMEs industriales \cite{Rodeiro2012}}

Este artículo respondió a la necesidad de pequeñas y medianas empresas industriales de mantenerse competitivas frente al avance tecnológico. Su propósito fue identificar factores clave para la adopción de tecnologías digitales en entornos de bajos recursos. La solución planteó estrategias prácticas de implementación con bajo costo y alto impacto. Los resultados mostraron un aumento en la productividad y eficiencia operativa.

\subsection{Sistemas de gestión de inventario en el sector de herramientas \cite{Flores2015}}

Esta tesis de maestría se enfocó en resolver los problemas de descontrol de stock en una cooperativa ferretera. El objetivo fue implementar un sistema digital de inventario para mejorar la visibilidad de existencias. La solución fue un software que permitía el registro en tiempo real de entradas y salidas, así como alertas por niveles críticos. Se logró reducir pérdidas y mejorar el control de recursos físicos.

\subsection{Importancia de la comunicación interna digitalizada \cite{Reyes2012}}

Este proyecto investigó la falta de coordinación entre departamentos en empresas medianas debido a comunicaciones fragmentadas. Se propuso implementar plataformas digitales de mensajería interna para mejorar la eficiencia organizacional. El estudio concluyó que la digitalización de la comunicación impacta positivamente en la productividad y en el ambiente laboral.

\subsection{Mejora de la experiencia del cliente mediante sistemas integrados \cite{Patino2019}}

Esta investigación abordó la desconexión entre áreas de atención al cliente y los procesos internos de las empresas. Su objetivo fue identificar cómo los sistemas digitales integrados pueden mejorar la experiencia del usuario. Se desarrolló una plataforma que unificaba seguimiento de servicios, notificaciones y canales de contacto. Como resultado, se observó un aumento en la fidelización y satisfacción del cliente.

\begin{longtable}{m{.05\paperwidth} *{2}{m{.33\paperwidth}} @{}}
	\caption{Comparación cualitativa de los trabajos relacionados con el proyecto.}
	\label{table:trabajosRelacionados}\\
	\hline
	\textbf{Ref.} & \textbf{Similitudes} & \textbf{Diferencias} \\
	\hline
	\endfirsthead
	
	\multicolumn{3}{c}{\textbf{Continuación de la Tabla \ref{table:trabajosRelacionados}}} \\
	\hline
	\textbf{Ref.} & \textbf{Similitudes} & \textbf{Diferencias} \\
	\hline
	\endhead
	\hline
	\endlastfoot
	
	\cite{Garcia07} &
	\begin{itemize}[topsep=0pt,itemsep=0pt,parsep=0pt,partopsep=0pt,leftmargin=*]
		\item Digitalización de procesos internos.
		\item Enfoque en eficiencia operativa.
		\item Sector industrial como contexto principal.
	\end{itemize} &
	\begin{itemize}[topsep=0pt,itemsep=0pt,parsep=0pt,partopsep=0pt,leftmargin=*]
		\item Falta de especialización en servicios técnicos.
	\end{itemize} \\
	\midrule
	
	\cite{Nin1992} &
	\begin{itemize}[topsep=0pt,itemsep=0pt,parsep=0pt,partopsep=0pt,leftmargin=*]
		\item Automatización del flujo técnico.
		\item Mejora de trazabilidad.
		\item Procesos estructurados por etapas.
	\end{itemize} &
	\begin{itemize}[topsep=0pt,itemsep=0pt,parsep=0pt,partopsep=0pt,leftmargin=*]
		\item Tecnología desactualizada.
		\item Poca atención al cliente final.
	\end{itemize} \\
	\midrule
	
	\cite{Rodeiro2012} &
	\begin{itemize}[topsep=0pt,itemsep=0pt,parsep=0pt,partopsep=0pt,leftmargin=*]
		\item Contexto PyME industrial.
		\item Adopción tecnológica gradual.
	\end{itemize} &
	\begin{itemize}[topsep=0pt,itemsep=0pt,parsep=0pt,partopsep=0pt,leftmargin=*]
		\item Escaso detalle técnico en implementación.
	\end{itemize} \\
	\midrule
	
	\cite{Flores2015} &
	\begin{itemize}[topsep=0pt,itemsep=0pt,parsep=0pt,partopsep=0pt,leftmargin=*]
		\item Gestión de inventario digital.
		\item Sector ferretero similar.
	\end{itemize} &
	\begin{itemize}[topsep=0pt,itemsep=0pt,parsep=0pt,partopsep=0pt,leftmargin=*]
		\item Alcance limitado al control de stock.
	\end{itemize} \\
	\midrule
	
	\cite{Reyes2012} &
	\begin{itemize}[topsep=0pt,itemsep=0pt,parsep=0pt,partopsep=0pt,leftmargin=*]
		\item Necesidad de comunicación interna efectiva.
		\item Uso de plataformas digitales organizacionales.
	\end{itemize} &
	\begin{itemize}[topsep=0pt,itemsep=0pt,parsep=0pt,partopsep=0pt,leftmargin=*]
		\item No implementa una solución tecnológica específica.
	\end{itemize} \\
	\midrule
	
	\cite{Patino2019} &
	\begin{itemize}[topsep=0pt,itemsep=0pt,parsep=0pt,partopsep=0pt,leftmargin=*]
		\item Mejora de experiencia del cliente.
		\item Seguimiento digital de servicios.
	\end{itemize} &
	\begin{itemize}[topsep=0pt,itemsep=0pt,parsep=0pt,partopsep=0pt,leftmargin=*]
		\item Contexto distinto (empresa propia).
	\end{itemize} \\
	\bottomrule
	\end{longtable}
	

	\section{Descripción técnica}

	El sistema propuesto para AironTools es una aplicación web empresarial compuesta por varios módulos funcionales que permiten automatizar y centralizar procesos clave como la atención al cliente, la gestión de servicios técnicos, el control de inventario y la comunicación interna. Estos módulos se integran mediante una arquitectura escalable que permite la interacción eficiente entre las distintas áreas operativas de la empresa.
	
	\subsection*{Arquitectura general del sistema}
	
	El sistema sigue un enfoque modular basado en microservicios, donde cada módulo funciona de forma autónoma pero coordinada mediante APIs RESTful. Esta arquitectura facilita el mantenimiento, la escalabilidad y la integración de nuevos componentes.
	
	La Figura~\ref{fig:arquitectura} muestra una vista general de la arquitectura del sistema, con sus principales componentes e interacciones.
	
	\begin{figure}[H]
		\centering
		\includegraphics[width=0.9\textwidth]{sistema.png}
		\caption{Arquitectura general del sistema propuesto para AironTools.}
		\label{fig:arquitectura}
	\end{figure}
	
	\subsection*{Módulo de usuarios y roles}
	
	Este módulo gestiona el acceso al sistema. Permite autenticar usuarios y asignarles distintos roles, como administrador, técnico, ventas o soporte. Su función es garantizar la seguridad del sistema y el acceso segmentado según responsabilidades.
	
	\subsection*{Módulo de clientes}
	
	Este módulo permite el registro de clientes, así como la consulta de su historial de servicios y datos de contacto. Facilita el seguimiento de cada cliente y mejora la experiencia de atención mediante un acceso rápido y centralizado a su información.
	
	\subsection*{Módulo de servicios técnicos}
	
	Administra todo el flujo operativo relacionado con los servicios que presta AironTools, como la reparación de herramientas. Incluye el registro del equipo, análisis técnico, cotización, reparación, entrega y recepción final. El flujo completo de este proceso se muestra en la Figura~\ref{fig:secuencia}.
	
	\begin{figure}[H]
		\centering
		\includegraphics[width=0.9\textwidth]{secuencia-sistema.png}
		\caption{Diagrama de secuencia que representa el flujo de atención a servicios técnicos.}
		\label{fig:secuencia}
	\end{figure}
	
	Además, los principales casos de uso del sistema, asociados al módulo de servicios técnicos, se ilustran en la Figura~\ref{fig:casosuso}.
	
	\begin{figure}[H]
		\centering
		\includegraphics[width=0.9\textwidth]{casos-uso-sistema.png}
		\caption{Diagrama de casos de uso del sistema.}
		\label{fig:casosuso}
	\end{figure}
	
	\subsection*{Módulo de inventario}
	
	Este módulo controla las existencias de productos, herramientas y refacciones. Permite registrar entradas, salidas y movimientos de stock, y genera alertas automáticas cuando se detectan niveles críticos. Su objetivo es prevenir faltantes y optimizar la gestión de recursos físicos.
	
	\subsection*{Módulo de notificaciones y comunicación interna}
	
	Incorpora funciones de mensajería interna y notificaciones automáticas que informan tanto a empleados como a clientes sobre el estado de sus solicitudes. Mejora la coordinación entre áreas y proporciona una atención más transparente al cliente.
	
	\subsection*{Resumen visual de módulos e interacciones}
	
	La Figura~\ref{fig:modulos} muestra un esquema con los módulos del sistema y sus interacciones, evidenciando cómo se comunican y complementan entre sí para cumplir con los objetivos operativos.
	
	\begin{figure}[H]
		\centering
		\includegraphics[width=0.9\textwidth]{componentes-sistema.png}
		\caption{Módulos principales del sistema de gestión y sus interacciones.}
		\label{fig:modulos}
	\end{figure}
	
	\section{Especificación técnica}

	El sistema de gestión empresarial para AironTools será accesible mediante una aplicación web y estará conformado por módulos que operan de forma integrada. El proyecto contempla el desarrollo de funcionalidades específicas orientadas a los procesos clave de la empresa: atención al cliente, gestión de servicios técnicos, control de inventario, comunicación interna y administración de usuarios.
	
	El desarrollo del sistema utilizará los siguientes elementos tecnológicos:
	\begin{itemize}
		\item \textbf{Lenguaje de programación y entorno:} TypeScript sobre Node.js para el backend; React.js con TypeScript para el frontend.
		\item \textbf{Base de datos:} MongoDB como sistema de almacenamiento de datos.
		\item \textbf{Interfaz de usuario:} HTML5, CSS3, Figma y Material UI.
		\item \textbf{Arquitectura:} Cliente-servidor basada en microservicios con comunicación vía APIs RESTful.
		\item \textbf{Control de versiones e integración continua:} Git y GitHub Actions.
		\item \textbf{Contenedores y despliegue:} Docker para contenerización de servicios y entornos de pruebas.
	\end{itemize}
	
	El sistema contempla como alcance mínimo el desarrollo completo de los siguientes módulos funcionales:
	
	\begin{itemize}
		\item Gestión de empleados y control de acceso mediante roles.
		\item Registro de clientes, seguimiento de historial y solicitudes.
		\item Flujo completo de servicios técnicos con notificaciones automatizadas.
		\item Control de inventario con registros de entradas, salidas y alertas por niveles críticos.
		\item Comunicación interna por mensajería o correo empresarial.
		\item Generación de reportes operativos para análisis y toma de decisiones.
	\end{itemize}
	
	Cada módulo será considerado finalizado cuando cumpla con los casos de uso definidos, haya sido probado funcionalmente y esté debidamente documentado. Además, el sistema deberá encontrarse desplegado en un entorno de pruebas funcional para su demostración.
	
	La Figura~\ref{fig:modulos} muestra un esquema general de los módulos definidos, y la Figura~\ref{fig:casosuso} representa los principales casos de uso considerados para la validación del sistema.
	
	\vspace{0.5cm}
	
	%Este texto SÍ debe incluirse para que la propuesta pueda ser aceptada.
	Al concluir el proyecto de integración se entregará a la Coordinación de Estudios de Ingeniería en Computación una carpeta digital que incluirá el reporte final del proyecto en un archivo PDF (sin restricciones)\footnote{Debe poder visualizarse sin solicitar contraseña}, el código fuente del proyecto en un archivo comprimido (sin restricciones)\footnote{Debe poder descomprimirse sin solicitar contraseña}. Además, la sección de apéndices del reporte final contendrá al menos un listado del código fuente desarrollado.


	\section{Calendario de actividades}

Las actividades a realizar durante el Trimestre 2025-Invierno en la UEA Proyecto de Integración de Ingeniería en Computación I (clave 1100113), con un valor de 18 créditos y una duración total de 198 horas, se presentan en la Tabla~\ref{table:calendarioActividades}.

\begin{longtable}{p{0.05\textwidth} p{0.4\textwidth} p{0.1\textwidth} p{0.35\textwidth}}
	\caption{Listado de actividades a realizar durante el Trimestre 2025-Invierno.}
  	\label{table:calendarioActividades}\\
	\toprule
	\textbf{No.} & \textbf{Actividad} & \textbf{Horas} & \textbf{Entregable} \\
	\hline
	\endfirsthead

	\multicolumn{4}{c}{\textbf{Continuación de la Tabla \ref{table:calendarioActividades}}}\\
	\hline
	\textbf{No.} & \textbf{Actividad} & \textbf{Horas} & \textbf{Entregable} \\
	\hline
	\endhead

	\hline
	\endlastfoot

	1 & Levantamiento de requerimientos y análisis del sistema actual en AironTools. & 20 & Documento de requerimientos \\
	\midrule

	2 & Diseño de arquitectura del sistema y definición de módulos. & 20 & Diagramas de arquitectura y diseño técnico \\
	\midrule

	3 & Desarrollo del módulo de autenticación y gestión de usuarios. & 25 & Módulo funcional con control de acceso y roles \\
	\midrule

	4 & Desarrollo del módulo de gestión de clientes. & 20 & Registro, historial y vista de clientes implementados \\
	\midrule

	5 & Desarrollo del módulo de servicios técnicos (flujo completo). & 25 & Módulo de flujo de servicio técnico automatizado \\
	\midrule

	6 & Desarrollo del módulo de inventario. & 20 & Registro, movimientos y alertas de inventario \\
	\midrule

	7 & Implementación de sistema de notificaciones y comunicación interna. & 20 & Notificaciones por correo, alertas internas y tareas asignadas \\
	\midrule

	8 & Pruebas funcionales e integración de los módulos. & 20 & Reporte de pruebas, casos de uso validados \\
	\midrule

	9 & Despliegue en entorno de pruebas y revisión técnica. & 15 & Sistema desplegado en servidor y documentación preliminar \\
	\midrule

	10 & Documentación técnica y elaboración de manual de usuario. & 13 & Manual de usuario, guía de instalación y documentación del sistema \\
	\bottomrule
\end{longtable}

\footnotetext{La planeación cubre el análisis, diseño, desarrollo, pruebas, despliegue y documentación del sistema, abarcando las 198 horas correspondientes a la UEA mencionada.}




\section{Factibilidad técnica y operativa}

\subsection{Factibilidad técnica}

El proyecto de desarrollo del sistema de gestión empresarial para AironTools es técnicamente viable, ya que el responsable del desarrollo cuenta con los conocimientos y habilidades necesarios para implementar los módulos definidos en el tiempo estipulado. Entre las competencias destacadas se encuentran:

\begin{itemize}
	\item Conocimientos avanzados en desarrollo web fullstack utilizando tecnologías como React.js, TypeScript y NestJS.
	\item Experiencia en diseño e implementación de arquitecturas modulares y sistemas empresariales.
	\item Habilidad en pruebas, documentación técnica y despliegue de sistemas.
\end{itemize}

Los recursos disponibles para el desarrollo y pruebas del sistema son los siguientes:

\begin{itemize}
	\item Equipos de cómputo con procesadores Intel Core i7, 16 GB de RAM y almacenamiento SSD.
	\item Acceso a entornos locales y servidores remotos para pruebas y despliegue del sistema.
	\item Conectividad de red estable para realizar pruebas multiusuario y simulaciones reales.
\end{itemize}

Las herramientas que se utilizarán durante el desarrollo incluyen:

\begin{itemize}
	\item \textbf{Frontend:} React.js con TypeScript, Figma para prototipado, y Material UI para diseño de interfaz.
	\item \textbf{Backend:} NestJS con Node.js y MongoDB como base de datos.
	\item \textbf{Control de versiones:} Git y GitHub.
	\item \textbf{Contenedores y despliegue:} Docker, GitHub Actions para integración y entrega continua.
\end{itemize}

No se requieren recursos físicos adicionales ni licencias de software especiales, ya que todas las herramientas son de uso libre o ya están disponibles.

\subsection{Factibilidad operativa}

El sistema propuesto presenta una alta factibilidad operativa dentro de la empresa AironTools, por las siguientes razones:

\begin{itemize}
	\item \textbf{Adaptabilidad a los procesos actuales:} El sistema está alineado con los flujos de trabajo reales de la empresa, por lo que su adopción no requerirá una reestructuración significativa.
	
	\item \textbf{Aceptación organizacional:} El proyecto cuenta con el respaldo directo del jefe de área, el Ing. Víctor Benjamín Aguilar Orocio, responsable de la implementación en la empresa.
	
	\item \textbf{Capacitación del personal:} Se prevé una estrategia de introducción y formación progresiva, asegurando que los empleados comprendan y utilicen el sistema eficazmente.
	
	\item \textbf{Facilidad de uso y soporte:} La interfaz estará diseñada con criterios de usabilidad, y el sistema incluirá funciones de ayuda y comunicación interna que facilitarán su soporte continuo.
	
	\item \textbf{Extensibilidad y mantenimiento:} Gracias a su arquitectura modular, el sistema podrá adaptarse a futuras necesidades o integrarse con nuevas tecnologías sin comprometer su estabilidad.
\end{itemize}

\vspace{0.5cm}
\section{Estimación de costos}

La presente sección muestra una estimación comercial del capital necesario para el desarrollo del sistema de gestión empresarial propuesto, considerando tanto los recursos técnicos como el valor del trabajo intelectual involucrado. Se han incluido costos asociados a infraestructura, herramientas, servicios y personal. Esta estimación representa una aproximación realista en caso de que el proyecto se desee escalar o comercializar.

La estimación de costos del proyecto se presenta en la Tabla~\ref{table:tablaCostos}.

\begin{longtable}{m{6.5cm} m{4.5cm} m{4cm}}
	\caption{Estimación de costos del proyecto.}
  	\label{table:tablaCostos}\\
  	\toprule
	\textbf{Descripción} & \textbf{Costo unitario (MXN)} & \textbf{Costo total (MXN)} \\
	\hline
	\endfirsthead

	\multicolumn{3}{c}{\textbf{Continuación de la Tabla \ref{table:tablaCostos}}}\\
	\hline
	\textbf{Descripción} & \textbf{Costo unitario (MXN)} & \textbf{Costo total (MXN)} \\
	\hline
	\endhead

	\hline
	\endlastfoot

Infraestructura de servidores (entorno de pruebas y producción) & \$3,000.00 $\times$ 1 mes & \$9,000.00 (3 meses) \\
\midrule

Servicios en la nube (almacenamiento, correos, CI/CD) & \$2,000.00 $\times$ 1 mes & \$6,000.00 (3 meses) \\
\midrule

Trabajo de desarrollo (fullstack)~\cite{Desarrollador} & \$25,000.00 $\times$ 1 mes & \$75,000.00 (3 meses) \\
\midrule

Diseño UI/UX y prototipado de interfaces & \$10,000.00 $\times$ 1 mes & \$30,000.00 (3 meses) \\
\midrule

Capacitación y soporte técnico interno & \$5,000.00 $\times$ 1 mes & \$15,000.00 (3 meses) \\
\midrule

Pruebas, validación y documentación & \$4,000.00 $\times$ 1 mes & \$12,000.00 (3 meses) \\
\midrule

\textbf{Costo total estimado} & — & \textbf{\$147,000.00} \\
\bottomrule
\end{longtable}

\footnotetext{Los valores fueron obtenidos con base en referencias de mercado, experiencia previa en proyectos similares y validación con el responsable directo en la empresa. Incluyen costos de infraestructura, servicios, diseño, desarrollo y soporte.}
